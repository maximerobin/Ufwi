\documentclass[12pt]{report}

\usepackage[utf8]{inputenc}
\usepackage[french]{babel}
\usepackage[T1]{fontenc}
\usepackage{longtable}
\usepackage{graphicx}

%-------------------------------------------------------------------------------------------------------------------------------
% Marges
\usepackage[top=2cm, bottom=2cm, left=2cm, right=2cm]{geometry}
%-------------------------------------------------------------------------------------------------------------------------------

%-------------------------------------------------------------------------------------------------------------------------------
% Entete et pied de page
\usepackage{fancyhdr} 
\fancypagestyle{plain}{%
	\fancyhf{} 
	\fancyhead{}                  
	\fancyfoot{}   
	\fancyfoot[L]{Ufwi}
	\fancyfoot[C]{\thepage}
	\fancyfoot[R]{\today}
	\chead{Projet tuteuré – Ufwi}
	\renewcommand{\headrulewidth}{1pt}   
	\renewcommand{\footrulewidth}{1pt}      
}
\pagestyle{plain}
%-------------------------------------------------------------------------------------------------------------------------------




%-------------------------------------------------------------------------------------------------------------------------------
% Gestion de l'affichage des chapitres
\makeatletter
\renewcommand{\@chapapp}{}
\makeatother
%-------------------------------------------------------------------------------------------------------------------------------

\title{Ufwi}
\author{Maxime Robin, Cyril Pierré, Valentin Frolich, Simon Barotte}
\date{\today}

\begin{document}



%-------------------------------------------------------------------------------------------------------------------------------
% Page d'accueil
\thispagestyle{empty}
\begin{center}
Licence Professionnel ASRALL


Projet tuteuré.

\vspace{2,5cm}
\textbf{\Huge Ufwi}


\end{center}


%-------------------------------------------------------------------------------------------------------------------------------

\newpage

%-------------------------------------------------------------------------------------------------------------------------------
% Sommaire
\renewcommand{\contentsname}{Sommaire}
\tableofcontents
%-------------------------------------------------------------------------------------------------------------------------------

\newpage

%-------------------------------------------------------------------------------------------------------------------------------
% Content
\chapter{Introduction}
Début du doc de Ufwi

%-------------------------------------------------------------------------------------------------------------------------------
\chapter{Liste des solutions de pare-feu par identification}
Les pare-feu par identification les plus connus : 
  \begin{itemize}
    \item AuthPF : Fonctionne sous OpenBSD et qui se repose sur SSH pour l'identification des utilisateurs : http://www.openbsd.org/faq/pf/authpf.html
    \item NuFW : projet ayant donné naissance à UFWI suite à la liquiditation de l'éditeur "EdenWall Technologies"
    \item Cyberoam : pare-feu entièrement basé sur l'identification, en utilisant une corrélation entre adresse MAC et utilisateur : http://www.cyberoam.com/fr/firewall.html
    \item CheckPoint (NAC Blade) : utilisation des règles de filtrage en fonction d'une authentifcation basée sur Kerberos, l'identité de son poste et du niveau de sécurité du poste ( mise à jour de sécurité / antivirus ) : http://www.cyberoam.com/fr/firewall.html
  \end{itemize}
%-------------------------------------------------------------------------------------------------------------------------------
\chapter{Externalisation des logs dans une BD MySQL}
\underline{Configuration du serveur BD}

Installation des paquets :

apt-get install apache2 php5 mysql-server nulog

\underline{Configuration de la passerelle : }

Configuration IP :

ifconfig eth0 192.168.1.137/24
ifconfig eth1 172.20.8.1/24

Installation des paquets :

apt-get install ulogd ulogd-mysql

Correction d'un bug : ajout d’une ligne dans le script de démarrage qui va charger un module

nano /etc/init.d/ulogd
export LD_PRELOAD=/usr/lib/libmysqlclient.so.16

Configuration de ulogd : modification de son fichier de configuration

nano /etc/ulogd.conf

Décommenter la ligne 46 (pour charger un module supplémentaire)

Renseigner les informations de connexion à la base de données :

paragraphe « [MYSQL] » ligne 59 :
table=’’ulog’’
pass=’’passulog’’
user=’’ulog’’
db=’’ulog’’
host=’’172.20.8.2’’

Configuration du serveur de BD:

Configuration IP :

ifconfig eth0 172.20.8.2/24

Lister tous les fichiers installés à l’installation de nulog :

dpkg –L nulog | more

Ouvrir le fichier suvant (démarche à suivre pour creer les tables de la base de données)

nano /usr/share/doc/nulog/README.Debian

Connexion à la base de données et création de l’utilisateur (les deux programmes vont se connecter avec ce compte):

mysql –u root –p
create database ulog;
create user 'ulog'@'\%' identified by 'passulog' ;
grant all privileges on ulog.* to ulog;
exit

Commandes de création de la base :

cd /usr/share/doc/nulog/scripts
gunzip ipv4.sql.gz
cat ipv4.sql | mysql –uulog –p ulog

Modification du fichier de configuration de mysql

nano /etc/mysql//my.cnf
ligne 47

Il faut qu’il écoute sur l’interface 172.20.8.2

bind address= ‘’172.20.8.2’’

Renommer les fichiers de configuration :

cd /etc/nulog
cp default. core.conf core.conf
cp default.nulog.conf nulog.conf
cp default.wrapper.conf wrapper.conf

Renseigner les informations de connexion à la base de données :

nano core.conf
host=localhost
db=ulog
user=ulog
password=passulog
table=ulog

Prise en compte des changements : redémarrage de services
Sur la passerelle :

/etc/init.d/ulogd restart

Sur le serveur :

/etc/init.d/ulogd restart


On choisit ce que l’on veut loguer avec iptables

\end{document}
