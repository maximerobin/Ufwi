\documentclass[12pt]{report}

\usepackage[utf8]{inputenc}
\usepackage[french]{babel}
\usepackage[T1]{fontenc}
\usepackage{longtable}
\usepackage{graphicx}

%-------------------------------------------------------------------------------------------------------------------------------
% Marges
\usepackage[top=2cm, bottom=2cm, left=2cm, right=2cm]{geometry}
%-------------------------------------------------------------------------------------------------------------------------------

%-------------------------------------------------------------------------------------------------------------------------------
% Entete et pied de page
\usepackage{fancyhdr} 
\fancypagestyle{plain}{%
	\fancyhf{} 
	\fancyhead{}                  
	\fancyfoot{}   
	\fancyfoot[L]{Ufwi}
	\fancyfoot[C]{\thepage}
	\fancyfoot[R]{\today}
	\chead{Projet tuteuré – Ufwi}
	\renewcommand{\headrulewidth}{1pt}   
	\renewcommand{\footrulewidth}{1pt}      
}
\pagestyle{plain}
%-------------------------------------------------------------------------------------------------------------------------------




%-------------------------------------------------------------------------------------------------------------------------------
% Gestion de l'affichage des chapitres
\makeatletter
\renewcommand{\@chapapp}{}
\makeatother
%-------------------------------------------------------------------------------------------------------------------------------

\title{Ufwi}
\author{Maxime Robin, Cyril Pierré, Valentin Frolich, Simon Barotte}
\date{\today}

\begin{document}



%-------------------------------------------------------------------------------------------------------------------------------
% Page d'accueil
\thispagestyle{empty}
\begin{center}
Licence Professionnel ASRALL


Projet tuteuré.

\vspace{2,5cm}
\textbf{\Huge Ufwi}


\end{center}


%-------------------------------------------------------------------------------------------------------------------------------

\newpage

%-------------------------------------------------------------------------------------------------------------------------------
% Sommaire
\renewcommand{\contentsname}{Sommaire}
\tableofcontents
%-------------------------------------------------------------------------------------------------------------------------------

\newpage

%-------------------------------------------------------------------------------------------------------------------------------
% Content
\chapter{Introduction}
Début du doc de Ufwi

%-------------------------------------------------------------------------------------------------------------------------------
\chapter{Liste des solutions de pare-feu par identification}
Les pare-feu par identification les plus connus : 
  \begin{itemize}
    \item AuthPF : Fonctionne sous OpenBSD et qui se repose sur SSH pour l'identification des utilisateurs : http://www.openbsd.org/faq/pf/authpf.html
    \item NuFW : projet ayant donné naissance à UFWI suite à la liquiditation de l'éditeur "EdenWall Technologies"
    \item Cyberoam : pare-feu entièrement basé sur l'identification, en utilisant une corrélation entre adresse MAC et utilisateur : http://www.cyberoam.com/fr/firewall.html
    \item CheckPoint (NAC Blade) : utilisation des règles de filtrage en fonction d'une authentifcation basée sur Kerberos, l'identité de son poste et du niveau de sécurité du poste ( mise à jour de sécurité / antivirus ) : http://www.cyberoam.com/fr/firewall.html
  \end{itemize}


\end{document}
