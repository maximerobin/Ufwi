\documentclass[t,12pt]{beamer}

%liste des packages utilsé
\usepackage[T1]{fontenc}
\usepackage[french]{babel}
\usepackage[utf8]{inputenc}
\usepackage{graphicx}
\usepackage{pslatex}
%definition d'une couleur pour les titres
\definecolor{grisbleu}{rgb}{0,0,153}


%definition du thème
\useoutertheme[height=0pt,left]{sidebar}
\usecolortheme{seahorse}
\setbeamercolor*{titlelike}{parent=structure}
\useinnertheme{circles}
\setbeamertemplate{frametitle}[default][right]


% contenu de la page de titre
\title{Projet Youpi}
\subtitle{\tiny{Bras mechanisé}}
\author{Frolich Valentin}
\date{\oldstylenums{Juin 2010}}

\begin{document}

%------- page de titre --------
\frame{\titlepage}

% --------- Sommaire ---------

\begin{frame} 
	\begin{center}{\Large Somaire }\end{center}
	\tableofcontents[currentsection]       %genere un somaire automatique (avec les \section)
\end{frame} 

%-----page Objectif-----------
\section{Objectif}                                                    %utilisé pour la generation du sommaire
\begin{frame}                                                         %définit le debut de la page
    \begin{center}{\textcolor{grisbleu}{\Large Objectif}}\end{center} %centre et colore le titre
    \begin{itemize}                                                   %definie une liste
	\item Relever la courbe de courant traversant le moteur.
	\item Mettre au point un système de limitation de courant.	
	\item Alimenter les 4 phases en limitant le courant.
	\item Faire communiquer entre eux les moteurs.
\end{itemize}
\end{frame}                                                            %définit la fin de la page


\end {document}
